% more spces between paragraphs
\documentclass{proposalnsf}

% for diagrams
%\def\xcolorversion{2.00}
%\def\xkeyvalversion{1.8}

% experiement float package for placing images where they appear in latex code
\usepackage{float}

\usepackage[version=0.96]{pgf}
\usepackage{tikz}
\usetikzlibrary{arrows,shapes,snakes,automata,backgrounds,petri,positioning,shadows,trees}
%\tikzexternalize[
%    prefix=tikz/,
%    shell escape={-shell-escape\space-output-directory=build},
%]
\usepackage[latin1]{inputenc}

% to put tikz side-by-side 
\usepackage{caption}
\usepackage{subcaption}

% can change spacing between lines
\usepackage{setspace}


% See this file for a set of pre-defined journal abbreviations
%\input{journal-abbreviations.tex} 

%\newcommand{\degrees}{$\!\!$\char23$\!$}
\renewcommand{\refname}{\centerline{References cited}}

% This handles hanging indents for publications
\def\rrr#1\\{\par
\medskip\hbox{\vbox{\parindent=2em\hsize=6.12in
\hangindent=4em\hangafter=1#1}}}

\def\baselinestretch{1}

% added to visualize the borders
%\usepackage[showframe]{geometry}


\begin{document}

\begin{center}
    \begin{spacing}{1.75}
      {\Large{\bf Expanding Assisted Dying Accessibility to those with Severe Dementia}}\\*[3mm]
    \end{spacing}

Zachary Smith 
%More PI Names

\end{center}

\tableofcontents


\noindent

\noindent


\renewcommand{\thepage} {B--\arabic{page}}

\newpage

% reset page numbering to 1.  This is helpful, since the text can only
% be 15 pages, and reviewers will want to believe we've kept within
% those limits

\pagenumbering{arabic}
\renewcommand{\thepage} {D--\arabic{page}}

\newpage

%\centerline{\bf Results from Prior SBIR Support}
%
%\noindent
%{\bf Previous Award Title}
%{\it award number} (PI); dates, \$amount
%
%Research carried out ....


\noindent{\Large \bf PROJECT DESCRIPTION}

\section{Background / Introduction}
(or 'evidence suggests that' and cite it) Something I've heard repeatedly from older, mentally healthy people is that they wouldn't want to keep living if they had severe dementia. Then there are two end-of-life narratives that follow: their life ends without ever having severe dementia, or they get severe dementia and live with it for an indefinite amount of time until they die (usually not directly from dementia). \textbf{We need to build the business and legal processes to enable the end-of-life choice to not live with severe dementia.}
\\

Because a dementia-related wish to die is different in nature than a terminally-ill-induced wish to die, we need a different mindset and process.
Currently, the two main hurdles for somebody with severe dementia to take advantage of the physician-assisted dying law are that you need to be[1]:
\begin{itemize}
    \item{"mentally competent, i.e. capable of making and communicating your health care decisions"}
    \item{"diagnosed with a terminal illness that will, within reasonable medical judgment, lead to death within six months"}
\end{itemize}

\noindent{I'll assume the reasons for these requirements are that we want to ensure that:}
\begin{itemize}
    \item{this is the patient's choice}
    \item{they are not making a short-sighted decision, and there is a strong guarantee that their medical condition and quality of life will monotonically decline}
\end{itemize}

\noindent{The presumed reasons for these laws are consistant with the reasons somebody with severe dementia would want to take advantage of physician-assisted dying given:}
\begin{itemize}
    \item{while they were mentally competent they've expressed they wouldn't want to keep living if they ever have severe dementia}
    \item{we haven't found a cure for severe dementia}
\end{itemize}


\bigskip
\bigskip

Canada, Luxembourg, Switzerland, and the Netherlands have legalized some form of physician-assisted death and / or euthanasia for patients with severe dementia, though it's unclear how often it's been applied to that demographic.[2]
\\

In the US, there are two main logistical hurdles for people who wish to end their life once they have severe dementia:
\begin{itemize}
    \item{They're not able to make their own medical decisions}
    \item{There are no established business and legal procedures to follow}
\end{itemize}
    

Imagine that assisted dying is legal for people with severe dementia, but there is not a well-defined procedure for making the request and carrying out the request. Here are a few scenarios I could imagine:

\textbf{Scenario 1: Their children are involved}

    there is a parent who has adult children, and the parent expresses the wish to not continue living if they get severe dementia
    the parent starts to get mild dementia, realizes it, but they struggle navigating the business / legal processes to put it in place. Even if they manage to plan it, the children will rebel with the argument that their mildly demented parent can't make these decisions for themself anymore
    the children have some sort of medical power of attorney agreement over their parent. They know of their parents wish but can't come to a consensus of how or when to move forward, both logistically and emotionally

\textbf{Scenario 2: there is a spouse ...........}


\section{Outline of legal decisions to be made}
The logistics around an assisted death law for somebody with severe dementia are much more complicated than somebody with a six-month prognosis. As a basic framework, we need to define the following:
\begin{itemize}
    \item{when is the request made}

    \item{who is involved}
      \begin{itemize}
          \item{family (spouse, children, grandchildren)}
          \item{doctors}
          \item{emotional / psychological support}
          \item{legal support}
      \end{itemize}

  \item{agenda in the years that follow the request}
      \begin{itemize}
          \item{dementia status check-ins}
          \item{emotional check-ins}
          \item{legal check-ins (the laws could change in the 2 - 30 years between request and action)}
      \end{itemize}

  \item{how the request can be redacted}
      \begin{itemize}
          \item{by who}
          \item{on what timeline}
      \end{itemize}

\end{itemize}
 
    

        
        
\section{Nonprofit Business Plan}
\subsection{Overview}
The mission is to make assisted dying an option for those with severe dementia. To do that, we need to:
      \begin{itemize}
          \item{legalize the practice}
          \item{establish the systems to support the practice}
      \end{itemize}

The common way to pass a law like this would be to raise money through donations and advocate.

The main idea here is to take a different approach that leverages the position of the benefiting group to the advantage of the cause by bootstrapping that process with capital, labor, and enthusiasm. We plan to do this by providing preparatory legal, business, and emotional support services to those who wish to take advantage of an assisted dying law should they get severe dementia.

\subsection{Why now}
A nonprofit with the mission to legalize assisted dying for those with severe dementia is in a unique and privileged position. I can foresee the following independently each taking roughly 10 - 20 years to mature:

      \begin{itemize}
          \item{legal work to legalize assisted dying for people with severe dementia}
          \item{development and implementation of the systems and business infrastructure to support the practice}
          \item{culture to embrace the practice}
      \end{itemize}


This conveniently aligns with those in their 60's to 80's who share a common sentiment to not want to continue to live if they have severe dementia. This group of people, in 10 - 20 years, will be in a position where they have a higher chance of having some form of dementia. Now is a particularly unique time to act because the size of that age demographic is big (how do I say this? baby boomers).

These circumstances open up unique advocacy and fundraising advantages. Those who will directly benefit in 10 - 20 years can bootstrap this idea. They can be first clients, helping fund the mission, and those retired might want to contribute to the mission through paid or volunteer advocacy work.

And we can structure and align the nonprofit in such a way that the preparation services are immediately useful for them by providing legal and family / emotional support (the idea would be to perform the preperation services as if we knew it would be legal. This could be troublesome because if a law doesn't pass in time, then it might be emotionally harmful for clients).

\subsection{Timeline} 
      \begin{itemize}
        \item{year 1 - year 2: education + define policy and business models + townhalls}
        \item{year 3 - year 20: services + education + advocacy + policy and business model refinement}
        \item{year 15 - year 20: pass bill (what is the legal process timeline? Official submit, approval / denial, next steps, challenging outcomes, etc) (need to do research to build a better estimated timeline. Look at death with dignity, and other bills like marijuana)}
      \end{itemize}


In year 1 and 2, the dialog from the townhalls and education programs will inform the policy draft and initial business plan, and vice versa. In years 3-7, we'll implement the business plan and refine it along the way. The organization and mission should begin to take a more concrete shape and have better-defined legal and cultural goals. Performing services will be an iterative process, and once we start implementing the business model, we will learn things that will make us want to improve or pivot the services.

(I believe it will take roughly this long because it will support many more people and require more peripheral support systems compared to the terminally-ill assisted death model)

\subsection{Vibe (and why this vibe)}
Death and money are two strong forces. An organization like this could easily devolve into a business that chases inheritances and possibly loses sight of it's mission. This organization should be structured in a way where the organization, employees, and members are never financially, emotionally, or politically incentivised to influence individual decisions around assisted death.

The business plan for the provided services must be clear, commonly agreed upon, and held to that standard. The leadership, doctors, lawyers, and other service providers involved must be incentivised appropriately.

\bigskip
\noindent{Build a culture of...}
      \begin{itemize}
        \item{listening to older people - don't let the organization get dominated by career-ists}
        \item{community-driven organizational decisions}
        \item{checks and balances - ensure the service providers don't influence individual decisions}
      \end{itemize}


Some suggested practical guidelines:
      \begin{itemize}
        \item{explicitly forbid financial succession planning services}
        \item{flat fee or sliding scale pricing model for services- net worth or inheritance should never be involved}
        \item{no outside funding - align organization soley with customers, and never be incentivized to please, prioritize, or give weight to non-customer donors (this might be part of why lots of non-profits get watered-down)}
        \item{    maybe even, once services are provided, disallow donations outside of services. This would help to incentivize prioritizing clients, and not let clients buy power (I think large donors, even if well-intentioned, can disrupt power dynamics, goals, and trajectory)}
        \item{no investments (this might be disallowed for non-profits anyway)}
      \end{itemize}


These guidelines will all help to tightly couple service and mission, and incentivize the organization to provide the best service possible to clients both directly and through legal representation of the group. My fear here is that if lack of money is limiting growth or not enough to pay people appropriately, the organization might come up with clever ways to make more money from clients, but in ways that don't provide a lot of client value. I think this would mean that the guidelines are artificially limiting, and actually disservicing the mission and clients. The goal of these guidelines are to have the 'right' amount of money - enough for the mission to be successful, have traction and grow, but not so much that the wrong people are incentivized to lead, or that the organization choses to priorizite sources that give larger sums of money.

That said, I believe the organization will need to be funded primarily through donations until it's able to sustain itself through the services provided, hopefully no longer than a year or two.

\subsection{Defining policy and business model}
Much like other death-related medical procedures (is this true?), policy and business model will be necessarily coupled. I've outlined some of the questions that will need to be answered in the section 'Outline of legal decisions to be made'. In the end, the law will determine the necessary business infrastructure and processes, but until then, we have to think about and build the process that we're comfortable with. I believe making this a community-driven process will create the best outcome both legally and culturally. Ideally, the community gives input about what processes would be most helpful to them and their families. This inherently informs us of what processes should be legally required by law to ensure every assisted death is decided upon and carried out in a way that minimizes ambiguity and where everybody is in agreeance that this is what the patient wants. I also believe this communal agreeance on processes will help the law, which will necessitate the processes, get passed.

\subsection{Education and Townhalls}
Education events and events where we receive feedback will mostly be a blurred line until the proposed policy and business models are better defined and in practice. I think the first year might look like a large, crowd-sourced, organized brainstorming session with a clear goal. While I believe that this bootstrapped business model will be a fruitful path to some form of legalized assisted dying for those with severe dementia, there are a lot of policy and business details that would be better decided by a community of interested parties. Additionally, educating and empowering others to contribute will help people be part of the mission, contribute to a brighter future, and be invested in the outcome. It'll also increase the power behind the mission, both in numbers and passion. Lastly, even if I wrote the most optimal policy and submitted it today, I doubt it would pass. Historically, having a lot of motivated people behind a grassroots push for policy change has led to success (find source?).

\bigskip
\noindent{Proposed education topics:}
      \begin{itemize}
        \item{history of Death with Dignity}
        \item{history of assisted death}
        \begin{itemize}
          \item{old cultures that practice(d) it}
          \item{when and why forbidden}
          \item{where and how it's still practiced}
        \end{itemize}
        \item{what we know so far about what this law would look like}
        \item{history and timeline of similar laws}
        \item{how to talk about this with their families / to other people}
      \end{itemize}

\bigskip
\noindent{Proposed feedback topics:}
      \begin{itemize}
        \item{what they would want out of this (maybe prior to coming to a session, give them some info, then have them think about what they would want the process to look like for them)}
        \item{feedback about proposed process, or if they have new ideas for process}
        \item{feedback from family members}
        \item{feedback from religious groups}
        \item{feedback from lawyers}
        \item{feedback from doctors}
        \item{feedback from counselors}
        \item{feedback from all interested parties}
      \end{itemize}

\subsection{Services}


As I've explained above, the main idea here is to bootstrap the passing of an assisted dying policy for those with severe dementia by providing prepatory services for those that may want to take advantage of such policy in the future, should the law pass. Those services will likely fall into the folowing categories:

      \begin{itemize}
          \item{legal support: preparing the individual and their family. I'm not sure if this will be standardized, or if individuals will required tailored legal documents. If legal requirements around assisted dying are invariable across all people, this could also look like consultation, and not directly provide any legal work, but inform and encourage clients and families to be legally prepared and protected. In this case, this service might look like group education sessions instead of individualized service.}
          \item{medical check-ins: a schedule of dementia status check-ups. Could be once / decade or yearly, depending on the severity.}
          \item{emotional support: individual and family support for this course. This could also be a decision check-in, to ensure that the client still wants this at every step of the way, and that there is no pressure from family or others to participate or not}
      \end{itemize}

Other services might just be fostering and supporting a community for people to talk about this.

\bigskip

Ultimately, the community will decide what services they want to receive.

\bigskip

It'll also be important to understand the financial and peoplepower needed to be successful in this mission, and design a business in a way that ensures this success. Because most of the value from these services is not guaranteed, and it's received 10 - 20 years in the future, the payment for the services might be feel disportianately pricey. Maybe a better way to view funding for this mission is that we only take donations from those that are active participants in the community, and that the funding is provided equitably from all interested and dedicated parties.

Overall, I think this is the section that could benefit from much more thought. What are other examples of services that are paid up front, but value isn't delivered for 10 - 20 years? I don't think that necessarily has to be the case, because I think that we can provide immediate value, and I really think that's the key to success in any business, especially at the beginning. I think part of the immediate value proposition is the hope that, should you get severe dementia, you might have the option and be legally prepared to take advantage of it. Because of this idea, of getting people totally 'legally' prepared, prior to passing the law, is that if the law does get passed in the way intended, and those people meet the criteria of the law (through the legal prep work we've done), then they can take advantage of it. This really isn't much different than any 'maybe payoff' except, it's individually backed, and not backed through taxpayer money (space exploration, science).

\subsection{Advocacy}
I think the advocacy will look similar to advocacy around other major policy change, like LGBTQ rights or marijuana legalization. It's largely awareness and education. The difference here, is that it has the potential to be led by an older demographic, which stands out from other movements.

\subsection{Risks}
\subsubsection{Business Risk}
Is this a business plan that enough people will want to contribute enough resources to see the plan go through, even if the end goal is ultimately unsuccessful?

This is the risk that this hypothesis is wrong, or that we are unable to execute the plan.
This risk can continuously be assessed, and course-corrected as needed.

\subsubsection{Product Risk}
The product has the primary value proposition of legalized and accessible assisted death for the customer, and a few secondary value propositions in the form of receiving our legal, medical, and emotional services, the community that comes with working toward a shared goal, and a hope for a future that more people want.

I can't think of any other product where the primary value is only possibly delivered 5 - 20 years after payment. 
Only the secondary value propositions are received more immediately. 

This risk is distict from the business risk, and can only be retired or realized when the effort ceazes with either a positive or negative result. 



\newpage
\pagenumbering{arabic}
\renewcommand{\thepage} {E--\arabic{page}}

\bibliographystyle{plain} % We choose the "plain" reference style
\bibliography{refs} % Entries are in the refs.bib file


\newpage
\pagenumbering{arabic}
\renewcommand{\thepage} {G--\arabic{page}}

\end{document}
